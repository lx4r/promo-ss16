\documentclass{article}
% \usepackage{ngerman}
\usepackage[utf8]{inputenc}
\usepackage{enumerate}
\usepackage{courier}
% \usepackage{paralist}
% \usepackage{amsmath}
% \usepackage{amsfonts}
\usepackage{minted}
\usepackage{mdframed}

% compile with:
% pdflatex -shell-escape <filepath>

\definecolor{bg}{rgb}{0.85,0.95,0.85}

\title{Altklausur Bry 2012 Remastered für Programmierung und Modellierung 2016}
\author{Alexander Isenko}
\begin{document}

\maketitle

\begin{center}
\textit{Besprechung am 22.Juli 2016}
\end{center}

\section*{Aufgabe 1}
Hier wird nach Funktionen gefragt die ihr euch selber einfallen lassen könnt, total egal wie kompliziert
oder einfach sie sind, sie müssen lediglich den Bedingungen entsprechen.
\begin{enumerate} [a)]
    \setlength\itemsep{3em}
    \item  Definieren sie eine monomorphe Funktion (keine Typvariablen, nur feste Typen) \\

    \textit{Lösung:}
    \begin{mdframed}[backgroundcolor=bg]
        \begin{minted}[linenos]{haskell}
aufgabe1a :: Int -> Int
aufgabe1a x = x + 1
        \end{minted}
    \end{mdframed}

    \item Definieren sie eine polymorphe Funktion (mit Typvariablen)

        \begin{enumerate}[i)]
        \setlength\itemsep{3em}
            \item parametrisch polymorph  (keine Typvariablen vor ($=>$)) \\
                \begin{mdframed}[backgroundcolor=bg]
                    \begin{minted}[linenos]{haskell}
aufgabe1bi :: a -> a
aufgabe1bi x = x
                    \end{minted}
                \end{mdframed}
            \item ad-hoc polymorph       (mid. eine Typvariable vor ($=>$)) \\
                \begin{mdframed}[backgroundcolor=bg]
                    \begin{minted}[linenos]{haskell}
aufgabe1bii :: Num a => a -> a
aufgabe1bii x = x + x
                    \end{minted}
                \end{mdframed}
        \end{enumerate}

\newpage

    \item  Definieren sie eine \textit{gecurriete} Funktion (eine Funktion die ein Tupel nimmt, aber mit \texttt{curry} stattdessen die Argumente hintereinander akzeptiert)
        \begin{minted}[linenos]{haskell}
-- Hilfestellung:

   > :t curry
   curry :: ((a, b) -> c) -> a -> b -> c
        \end{minted}
        \hfill \\
        \textit{Lösung:}
        \begin{mdframed}[backgroundcolor=bg]
            \begin{minted}[linenos]{haskell}
aufgabe1c :: a -> b -> a
aufgabe1c a b = curry f a b
    where
        f :: (a,b) -> a
        f (a,b) = a
            \end{minted}
        \end{mdframed}
\end{enumerate}

\section*{Aufgabe 3}
Definieren sie die Funktion \texttt{reverse} für Listen in Haskell.
    \begin{minted}[linenos]{haskell}
        > :t reverse
        [a] -> [a]

        > reverse "hallo"
        "ollah"
        
        > reverse [1,2,3]
        [3,2,1]
        
        > reverse []
        []
    \end{minted}
    \hfill \\

    Lösung:
\begin{mdframed}[backgroundcolor=bg]
    \begin{minted}[linenos]{haskell}
aufgabe3 :: [a] -> [a]
aufgabe3 []     = []
aufgabe3 (x:xs) = aufgabe3 xs ++ [x]
    \end{minted}
\end{mdframed}

\newpage

\section*{Aufgabe 4}
Sei folgender Code gegeben, was ist das Ergebnis von \texttt{res}?
\begin{minted}[linenos]{haskell}
y = 5
x = 2
goo y     = x * y
fuu (x,y) = x + goo y
res = (goo y, fuu (5,7))
\end{minted}

\textit{Lösung:}
\begin{mdframed}[backgroundcolor=bg]
    \begin{minted}[linenos]{haskell}
    (goo y, fuu (5,7))    -- y = 5
 => (goo 5, fuu (5,7))    -- goo auswerten
 => (x * 5, fuu (5,7))    -- x = 2
 => (2 * 5, fuu (5,7))    -- 2 * 5 = 10
 => (10,    fuu (5,7))    -- fuu auswerten
 => (10,    5 + goo 7)    -- goo auswerten
 => (10,    5 + x * 7)    -- x = 2
 => (10,    5 + 2 * 7)    -- 5 + 2 * 7 = 19
 => (10, 19)
    \end{minted}
\end{mdframed}

\section*{Aufgabe 6}
Definieren sie das Standartskalarprodukt mithilfe von \texttt{zip} und \texttt{Data.List.foldl/foldr}

\begin{minted}[linenos]{haskell}
--  Typsignatur:
--
skalarProdukt :: Num a => [a] -> [a] -> a
--
--  Beispiel:
--
skalarProdukt [1,2,3] [4,5,6] = 1*4 + 2*5 + 3*6 = 4 + 10 + 18 = 32

-- Hilfestellung:
--
zip :: [a] -> [b] -> [(a,b)]
zip []     []     = []
zip (x:xs) (y:ys) = (x,y) : zip xs ys

foldl :: (b -> a -> b) -> b -> [a] -> b
foldl f acc []     = acc
foldl f acc (x:xs) = foldl f (f acc x) xs

foldr :: (a -> b -> b) -> b -> [a] -> b
foldr f acc []     = acc
foldr f acc (x:xs) = f x (foldr f acc xs)
\end{minted}

\textit{Lösung:}
\begin{mdframed}[backgroundcolor=bg]
    \begin{minted}[linenos, escapeinside=||]{haskell}
-- Da Addition kommutativ ist, können wir foldl
-- oder foldl benutzen
--
skalarProdukt :: Num a => [a] -> [a] -> a
skalarProdukt xs ys =
    foldl (||\||acc (x,y) -> x*y + acc) 0 (zip xs ys)

skalarProdukt' :: Num a => [a] -> [a] -> a
skalarProdukt' xs ys =
    foldr (||\||(x,y) acc  -> x*y + acc) 0 (zip xs ys)
    \end{minted}
\end{mdframed}

\section*{Aufgabe 7}
Sei folgender Code gegeben.
\begin{minted}[linenos, escapeinside=||]{haskell}
ks = [3,5,7]

pred = ||\||x -> x < 0

f p x (y:ys) = if p y
                   then f (not . p) x ys
                   else f p         x ys
f p x _      = p x

g x = let f x = x
      in g (f x)
\end{minted}

Geben sie wenn möglich den Wert bzw. den Typ der folgenden Ausdrücke an:

\newpage

\begin{enumerate} [a)]
    \setlength\itemsep{3em}
    \item \texttt{f} \\[2mm]
    \textit{Lösung:}
    \begin{mdframed}[backgroundcolor=bg]
        \begin{minted}[linenos, escapeinside=||]{haskell}
-- Aus Z.5 sieht man dass (p y) ein Bool zurückgibt,
-- weil es im Prädikat des if-then-else steht
p :: a -> Bool

-- 2. Aus Z.8 sehe ich dass die Funktion f als
-- Rückgabewert ein Bool hat, weil wir (p x) aufrufen
f :: (a -> Bool) -> ... -> ... -> Bool

-- 3. Aus Z.5 und Z.8 sieht man dass in 'p' 
-- sowohl 'x' als auch 'y' eingesetzt werden
-- => sie sind vom gleichen Typ

-- 4. Wir sehen aus Z.5 dass (y:ys) eine Liste ist.
f :: (a -> Bool) -> a -> [a] -> Bool
        \end{minted}
    \end{mdframed}
    \item \texttt{f pred 0 []} \\[2mm]
    \textit{Lösung:}
    \begin{mdframed}[backgroundcolor=bg]
        \begin{minted}[linenos, escapeinside=||]{haskell}
-- Da für 'f' alle Argumente befüllt sind ist der
-- Typ vom Rückgabewert
Bool
-- Das dritte Argument ist eine leere Liste,
-- wir kommen in den zweiten Fall von 'f', Z.8
pred 0 => 0 < 0 => False
        \end{minted}
    \end{mdframed}

\newpage

    \item \texttt{f pred 0 ks} \\[2mm]
    \textit{Lösung:}
    \begin{mdframed}[backgroundcolor=bg]
        \begin{minted}[linenos, escapeinside=||]{haskell}

-- Da für 'f' alle Argumente befüllt sind ist der
-- Typ vom Rückgabewert
Bool
-- 2. Das dritte Argument ist keine leere Liste,
-- wir kommen in den ersten Fall von 'f', Z.8

f pred 0 [3,5,7]
=> if pred 3     -- (3 < 0) => False
       then f (not . pred) 0 [5,7]
       else f (pred) 0 [5,7]

f pred 0 [5,7]    -- Erster Fall von 'f',
                  -- Liste ist nicht leer

=> if pred 5     -- (5 < 0) => False
      then f (not . pred) 0 [7]
      else f (pred) 0 [7]

f pred 0 [7]      -- Erster Fall von 'f',
                  -- Liste ist nicht leer

=> if pred 7      -- (7 < 0) => False
       then f (not . pred) 0 []
       else f (pred) 0 []

f pred 0 []       -- Zweiter Fall von 'f',
                  -- da Liste leer Z.8

=> pred 0 => 0 < 0 => False
        \end{minted}
    \end{mdframed}

\newpage

    \item \texttt{g} \\[2mm]
    \textit{Lösung:}
    \begin{mdframed}[backgroundcolor=bg]
        \begin{minted}[linenos, escapeinside=||]{haskell}
-- Wir sehen dass die Funktion 'f' die in 'g'
-- definiert wurde, lediglich ihr Argument zurückgibt
f :: a -> a    -- identisch zur Funktion id
id :: a -> a

-- Wir haben nur einen Fall, wo wir uns rekursiv
-- mit dem gleichen Argument immer wieder aufrufen.
-- Es gibt sonst keine Einschränkungen, der Typ ist
g :: a -> b
        \end{minted}
    \end{mdframed}

    \item \texttt{g ks} \\[2mm]
    \textit{Lösung:}
    \begin{mdframed}[backgroundcolor=bg]
        \begin{minted}[linenos, escapeinside=||]{haskell}

-- Da 'ks' eine Liste von Zahlen ist, ist sie vom Typ.
ks :: Num a => [a]
-- Da 'g' vom Typ 'a -> b' ist, setzen wir den Typ
-- von 'ks' für das Argument ein.
g ks :: b
-- Diese Funktion terminiert nicht, somit
-- gibt es kein Ergebnis
        \end{minted}
    \end{mdframed}
\end{enumerate}


\end{document}